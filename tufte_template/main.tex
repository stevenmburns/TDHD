\documentclass{tufte-book}

\hypersetup{colorlinks}% uncomment this line if you prefer colored hyperlinks (e.g., for onscreen viewing)

%%
% Book metadata
\title{Test-Driven Hardware Developement\\ \large
Techniques to Develop Better Quality Designs With Less Effort}
\author{Steve Burns}
\publisher{Second Son Publishing}

%%
% If they're installed, use Bergamo and Chantilly from www.fontsite.com.
% They're clones of Bembo and Gill Sans, respectively.
%\IfFileExists{bergamo.sty}{\usepackage[osf]{bergamo}}{}% Bembo
%\IfFileExists{chantill.sty}{\usepackage{chantill}}{}% Gill Sans

%\usepackage{microtype}

\usepackage{color}
\usepackage{fancyvrb}
\usepackage{comment}
\usepackage{minted}
\usepackage{xcolor} % to access the named colour LightGray
\definecolor{LightGray}{gray}{0.9}
%%

%%
% For nicely typeset tabular material
\usepackage{booktabs}

%%
% For graphics / images
\usepackage{graphicx}
\setkeys{Gin}{width=\linewidth,totalheight=\textheight,keepaspectratio}
\graphicspath{{graphics/}}

% The fancyvrb package lets us customize the formatting of verbatim
% environments.  We use a slightly smaller font.
\usepackage{fancyvrb}
\fvset{fontsize=\normalsize}

%%
% Prints argument within hanging parentheses (i.e., parentheses that take
% up no horizontal space).  Useful in tabular environments.
\newcommand{\hangp}[1]{\makebox[0pt][r]{(}#1\makebox[0pt][l]{)}}

%%
% Prints an asterisk that takes up no horizontal space.
% Useful in tabular environments.
\newcommand{\hangstar}{\makebox[0pt][l]{*}}

%%
% Prints a trailing space in a smart way.
\usepackage{xspace}

% Prints the month name (e.g., January) and the year (e.g., 2008)
\newcommand{\monthyear}{%
  \ifcase\month\or January\or February\or March\or April\or May\or June\or
  July\or August\or September\or October\or November\or
  December\fi\space\number\year
}


% Prints an epigraph and speaker in sans serif, all-caps type.
\newcommand{\openepigraph}[2]{%
  %\sffamily\fontsize{14}{16}\selectfont
  \begin{fullwidth}
  \sffamily\large
  \begin{doublespace}
  \noindent\allcaps{#1}\\% epigraph
  \noindent\allcaps{#2}% author
  \end{doublespace}
  \end{fullwidth}
}

% Inserts a blank page
\newcommand{\blankpage}{\newpage\hbox{}\thispagestyle{empty}\newpage}

\usepackage{units}

% Typesets the font size, leading, and measure in the form of 10/12x26 pc.
\newcommand{\measure}[3]{#1/#2$\times$\unit[#3]{pc}}

% Macros for typesetting the documentation
\newcommand{\hlred}[1]{\textcolor{Maroon}{#1}}% prints in red
\newcommand{\hangleft}[1]{\makebox[0pt][r]{#1}}
\newcommand{\hairsp}{\hspace{1pt}}% hair space
\newcommand{\hquad}{\hskip0.5em\relax}% half quad space
\newcommand{\TODO}{\textcolor{red}{\bf TODO!}\xspace}
\newcommand{\ie}{\textit{i.\hairsp{}e.}\xspace}
\newcommand{\eg}{\textit{e.\hairsp{}g.}\xspace}
\newcommand{\na}{\quad--}% used in tables for N/A cells
\providecommand{\XeLaTeX}{X\lower.5ex\hbox{\kern-0.15em\reflectbox{E}}\kern-0.1em\LaTeX}
\newcommand{\tXeLaTeX}{\XeLaTeX\index{XeLaTeX@\protect\XeLaTeX}}
% \index{\texttt{\textbackslash xyz}@\hangleft{\texttt{\textbackslash}}\texttt{xyz}}
\newcommand{\tuftebs}{\symbol{'134}}% a backslash in tt type in OT1/T1
\newcommand{\doccmdnoindex}[2][]{\texttt{\tuftebs#2}}% command name -- adds backslash automatically (and doesn't add cmd to the index)
\newcommand{\doccmddef}[2][]{%
  \hlred{\texttt{\tuftebs#2}}\label{cmd:#2}%
  \ifthenelse{\isempty{#1}}%
    {% add the command to the index
      \index{#2 command@\protect\hangleft{\texttt{\tuftebs}}\texttt{#2}}% command name
    }%
    {% add the command and package to the index
      \index{#2 command@\protect\hangleft{\texttt{\tuftebs}}\texttt{#2} (\texttt{#1} package)}% command name
      \index{#1 package@\texttt{#1} package}\index{packages!#1@\texttt{#1}}% package name
    }%
}% command name -- adds backslash automatically
\newcommand{\doccmd}[2][]{%
  \texttt{\tuftebs#2}%
  \ifthenelse{\isempty{#1}}%
    {% add the command to the index
      \index{#2 command@\protect\hangleft{\texttt{\tuftebs}}\texttt{#2}}% command name
    }%
    {% add the command and package to the index
      \index{#2 command@\protect\hangleft{\texttt{\tuftebs}}\texttt{#2} (\texttt{#1} package)}% command name
      \index{#1 package@\texttt{#1} package}\index{packages!#1@\texttt{#1}}% package name
    }%
}% command name -- adds backslash automatically
\newcommand{\docopt}[1]{\ensuremath{\langle}\textrm{\textit{#1}}\ensuremath{\rangle}}% optional command argument
\newcommand{\docarg}[1]{\textrm{\textit{#1}}}% (required) command argument
\newenvironment{docspec}{\begin{quotation}\ttfamily\parskip0pt\parindent0pt\ignorespaces}{\end{quotation}}% command specification environment
\newcommand{\docenv}[1]{\texttt{#1}\index{#1 environment@\texttt{#1} environment}\index{environments!#1@\texttt{#1}}}% environment name
\newcommand{\docenvdef}[1]{\hlred{\texttt{#1}}\label{env:#1}\index{#1 environment@\texttt{#1} environment}\index{environments!#1@\texttt{#1}}}% environment name
\newcommand{\docpkg}[1]{\texttt{#1}\index{#1 package@\texttt{#1} package}\index{packages!#1@\texttt{#1}}}% package name
\newcommand{\doccls}[1]{\texttt{#1}}% document class name
\newcommand{\docclsopt}[1]{\texttt{#1}\index{#1 class option@\texttt{#1} class option}\index{class options!#1@\texttt{#1}}}% document class option name
\newcommand{\docclsoptdef}[1]{\hlred{\texttt{#1}}\label{clsopt:#1}\index{#1 class option@\texttt{#1} class option}\index{class options!#1@\texttt{#1}}}% document class option name defined
\newcommand{\docmsg}[2]{\bigskip\begin{fullwidth}\noindent\ttfamily#1\end{fullwidth}\medskip\par\noindent#2}
\newcommand{\docfilehook}[2]{\texttt{#1}\index{file hooks!#2}\index{#1@\texttt{#1}}}
\newcommand{\doccounter}[1]{\texttt{#1}\index{#1 counter@\texttt{#1} counter}}

% Generates the index
\usepackage{makeidx}
\makeindex

\begin{document}

% Front matter
\frontmatter

% r.1 blank page
\blankpage

% v.2 epigraphs
\newpage\thispagestyle{empty}
\openepigraph{%
I'm not a great programmer; I'm just a good programmer with great habits.
}{Kent Beck% AZ Quotes
}
\vfill
\openepigraph{%
Make it work, make it right, make if fast.
}{Kent Beck}
\vfill
\openepigraph{%
If testings costs more than not testing, then don't test.
}{Kent Beck}


% r.3 full title page
\maketitle


% v.4 copyright page
\newpage
\begin{fullwidth}
~\vfill
\thispagestyle{empty}
\setlength{\parindent}{0pt}
\setlength{\parskip}{\baselineskip}
Copyright \copyright\ \the\year\ \thanklessauthor

\par\smallcaps{Published by \thanklesspublisher}

\par\smallcaps{Using templates from tufte-latex.github.io/tufte-latex}

\par Licensed under the Apache License, Version 2.0 (the ``License''); you may not
use this file except in compliance with the License. You may obtain a copy
of the License at \url{http://www.apache.org/licenses/LICENSE-2.0}. Unless
required by applicable law or agreed to in writing, software distributed
under the License is distributed on an \smallcaps{``AS IS'' BASIS, WITHOUT
WARRANTIES OR CONDITIONS OF ANY KIND}, either express or implied. See the
License for the specific language governing permissions and limitations
under the License.\index{license}

\par\textit{First printing, \monthyear}
\end{fullwidth}

% r.5 contents
\tableofcontents

\listoffigures

\listoftables

% r.7 dedication
\cleardoublepage
~\vfill
\begin{doublespace}
\noindent\fontsize{18}{22}\selectfont\itshape
\nohyphenation
Dedicated to those who want to construct correct hardware quickly
\end{doublespace}
\vfill
\vfill


% r.9 introduction
\cleardoublepage
\chapter*{Introduction}

This book provides many examples of how to design hardware using Agile development techniques proven in the software world. We will show how Test-Driven Development (TDD) methods can be applied to hardware design problems.

%%
% Start the main matter (normal chapters)
\mainmatter


\chapter{Test-Driven Software Development}
\label{ch:tdsd}

\newthought{Software} development is varies in difficulty depending on what you intend to implement. Test-driven development techniques (TDD) provide a structure to help developing software when what you're trying to do become complex. The main idea is to write tests as you develop the software. For a new feature you are adding, you add a test---make sure the test fails (ensuring you are testing something new), then add code to the implementation to make the test pass.

\vspace{-4\baselineskip}\footnotetext{Failing tests are called {\em red} and passing tests are called {\em green}. This two step process is then: 1) add a test and check that it is red; 2) change the implementation and make sure the test are green}
\vspace{4\baselineskip}

\newthought{We will start} our introduction to software TDD using the simple example of counting the number of bits set in an integer (restricted to eight bits.)
There are a number of algorithms for do this. We will start with some very straightforward ones and then move on to more efficient and subtle techniques.
We will do this all in Scala and use the Scalatest test framework to perform the testing. You can start by creating a simple test that just tests whether zero is returned if there are no bits set.

\newminted{scala}{frame=lines,framesep=2mm,baselinestretch=1.2,bgcolor=LightGray,fontsize=\footnotesize,linenos}

\newminted{bash}{frame=lines,framesep=2mm,baselinestretch=1.2,bgcolor=LightGray,fontsize=\footnotesize,linenos}

\begin{scalacode}
// src/test/scala/tdhd.scala
package tdhd
import org.scalatest.freespec.AnyFreeSpec

class PopCountSoftwareTest extends AnyFreeSpec {
  "Popcount with value 0 should return 0" in {
    assert(PopCount(0) == 0)
  }
}
\end{scalacode}
\vspace{-6\baselineskip}\footnotetext{%
Line 2: set the current package to {\tt tdhd};
Line 3: import the unit testing class from {\tt scalatest};
Line 5: define test using {\tt AnyFreeSpec} with an assertion in the body
}
\vspace{6\baselineskip}

We need some implementation for the code to compile:
\vspace{3\baselineskip}\footnotetext{%
Line 5: we deliberately return the wrong number here because we want to start out with a failing test.
}%
\vspace{-3\baselineskip}%
\begin{scalacode}
// src/main/scala/tdhd.scala
package tdhd
object PopCount {
  def apply(n : BigInt) : BigInt = {
    1
  }
}
\end{scalacode}
This is intended to fail. We want to show that the testing infrastructure is working and show this configuration (test and implementation) as red in the beginning. When we run the this using the command line:
\begin{bashcode}
sbt 'test'
\end{bashcode}
we get output that looks something like this:

\vspace{8\baselineskip}\footnotetext{%
The red text here indicates places that a test failed.
}
\vspace{-8\baselineskip}
\inputminted[style=ansi,fontsize=\scriptsize,linenos,bgcolor=LightGray,frame=lines,framesep=2mm,baselinestretch=1.2]{ansi-terminal}{../chisel/fail.ansi-txt}

We know that the test failed and can now change the implementation so that it works for this test.
\begin{scalacode}
// src/main/scala/tdhd.scala
package tdhd
object PopCount {
  def apply(n : BigInt) : BigInt = {
    0
  }
}
\end{scalacode}
This time the test should pass.

\vspace{5\baselineskip}\footnotetext{%
No red text here (all black, green, and cyan) indicates that all tests passed.
}\vspace{-5\baselineskip}%
\inputminted[style=ansi,fontsize=\scriptsize,linenos,bgcolor=LightGray,frame=lines,framesep=2mm,baselinestretch=1.2]{ansi-terminal}{../chisel/pass.ansi-txt}%
We aren't testing the full functionality yet, so we need to add more tests---one where the result is not zero should trigger a failure.
\begin{scalacode}
// src/test/scala/tdhd.scala
package tdhd
import org.scalatest.freespec.AnyFreeSpec

class PopCountSoftwareTest extends AnyFreeSpec {
  "Popcount with value 0 should return 0" in {
    assert(PopCount(0) == 0)
  }
  "Popcount with value 255 should return 8" in {
    assert(PopCount(255) == 8)
  }
}
\end{scalacode}
The test results are:
\vspace{7\baselineskip}\footnotetext{%
We have both passing and failing test here.
}\vspace{-7\baselineskip}%
\inputminted[style=ansi,fontsize=\scriptsize,linenos,bgcolor=LightGray,frame=lines,framesep=2mm,baselinestretch=1.2]{ansi-terminal}{../chisel/fail3.ansi-txt}%

At this point we need to improve our implementation again. We could add another case for all the bits being set, or we could try to implement a general algorithm. Here is a straightforward technique that walks through each bit position counting the number of true bits:
\vspace{3\baselineskip}\footnotetext{%
You can do this functionally---something like:
\mintinline[fontsize=\scriptsize]{scala}{((0 until 8) map {i => if(isset(i)) 1 else 0}).sum}
but I'm not convinced this is easier to read than the imperative program.
I frequently write things imperatively first, and then convert them to a functional program, and then think ``why did I do that?'', and convert it back. There is nothing wrong with mutable variables and imperative programs, even as we shall see later, in the definition of hardware.
}\vspace{-3\baselineskip}%
\begin{scalacode}
// src/main/scala/tdhd.scala
package tdhd
object PopCount {
  def isset(i: Int) = (n & (BigInt(1)<<i)) != BigInt(0)
  def apply(n : BigInt) : BigInt = {
    var sum : BigInt = 0
    for { i <- 0 until 8 } {
      if (isset(i))
        sum += 1
      }
    }
    sum
  }
}
\end{scalacode}
The helper function {\tt isset} is used to determine whether or not the ith bit is set in the input word {\tt n}. The mutable variable {\tt sum} counts the number of set bits.
If we run the test with this implementation, we get:
\inputminted[style=ansi,fontsize=\scriptsize,linenos,bgcolor=LightGray,frame=lines,framesep=2mm,baselinestretch=1.2]{ansi-terminal}{../chisel/pass_troll.ansi-txt}
Our test is green, but we are only testing a couple of cases. We should improve our test to include multiple cases.
Here is a possible {\em exhaustive} tester:
\begin{scalacode}
// src/test/scala/tdhd.scala
class PopCountExhaustiveSoftwareTest extends AnyFreeSpec {
  "Popcount should work for all values in the range [0,256)" in {
    for { i <- BigInt(0) until 256 } {
      assert(PopCount(i) == i.bitCount)
    }
  }
}
\end{scalacode}
which relies on the method {\tt bitCount} of {\tt BigInt} to provide a reference solution.
The result here, is green, and works fine.
\inputminted[style=ansi,fontsize=\scriptsize,linenos,bgcolor=LightGray,frame=lines,framesep=2mm,baselinestretch=1.2]{ansi-terminal}{../chisel/pass_exhaustive_troll.ansi-txt}


There are other techniques for counting the number of set bits in a word. The following bit manipulation trick clears iteratively clears the least significant set bit until there are no remaining bits.
\begin{scalacode}
// src/main/scala/tdhd.scala
package tdhd
object PopCount {
  def apply(n : BigInt) : BigInt = {
    var sum : BigInt = 0
    var x = n
    while (x != BigInt(0)) {
      // 01001111   (x-1)
      // 01010000   (x)
      // -------- &
      // 01000000   (new x)
      x = (x - 1) & x
      sum += 1
    }
    sum
  }
}
\end{scalacode}
Here, as long as the current word {\tt x} is not all zeroes, we subtract one from {\tt x} and then bitwise-AND that with {\tt x}. This clears the least significant set bit (because that bit is zero in {\tt x-1}) and leaves all bits to the left (and right) of that unchanged.

\section{Generalizations}
The type of problems we want to solve in this book usually have a straightforward solution, thay while correct, are deficit is some other attribute. Perhaps this solution is too slow or too big, so we want to develop techniques that allow use to try out multiple implementations and run the same test suite on these multiple implementations. We may also want to run different sized instances of the problems as well. To do this for how tests and implementations above, we want to create a new level of abstraction in the test framework, which will allow us to generate multiple {\em tests} per {\em tester}.
Here we can rewrite our two existing tests as testers:
\begin{scalacode}
// src/test/scala/tdhd.scala
class PopCountSoftwareTester(func: BigInt => BigInt)
  extends AnyFreeSpec {
  "Popcount with value 0 should return 0" in {
    assert(func(0) == 0)
  }
  "Popcount with value 255 should return 8" in {
    assert(func(255) == 8)
  }
}

class PopCountExhaustiveSoftwareTester(func: BigInt => BigInt)
  extends AnyFreeSpec {
  "Popcount should work for all values in the range [0,256)" in {
    for { i <- BigInt(0) until 256 } {
      assert(func(i) == i.bitCount)
    }
  }
}
\end{scalacode}

%%
% The back matter contains appendices, bibliographies, indices, glossaries, etc.

\backmatter

\bibliography{sample-handout}
\bibliographystyle{plainnat}


\printindex

\end{document}
