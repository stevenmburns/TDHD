%-------------------------------------------
%Introducing Sentence of a Paragraph: bold, capitals
%use \textbf

%-------------------------------------------
%Emphasis : italics or bold (only in rare cases)
% use \textit{}, \textbf

%-------------------------------------------
% Quotes : italics (sparingly)
% use \textit{}

%-------------------------------------------
% References to titles: italics. Check the preamble in main.tex




%-------------------------------------------
%BOOK NAME: bold + caps + italics
\newcommand{\bookname}[1]{\textbf{\textit{\uppercase{#1}}}}


%-------------------------------------------
%Titles: bold, capitals, different font, different color, initial caps
\definecolor{dark-blue}{RGB}{28, 69, 135}
\addtokomafont{section}{\color{dark-blue}\LARGE\sffamily\bfseries}
\addtokomafont{subsection}{\color{purple}\large\sffamily\bfseries}

% No header or footer on chapter pages
\renewcommand*{\chapterpagestyle}{empty}

% If you want to suppress chapter numbers from chapter and section titles for easier reading, uncomment these lines
%\renewcommand*\chapterformat{}
%\renewcommand{\thesection}{\arabic{section}}

\ifxetex
%-------------------------------------------
% Define space before and after a chapter, section, and subsection
\renewcommand*\chapterheadstartvskip{\vspace*{-3\topskip}}
\renewcommand*\chapterheadendvskip{\vskip-.5\baselineskip\noindent{\color{gray}\rule{\linewidth}{2pt}}\par}
\newcommand\subsectionprelude{\vspace{-\parskip}}
\newcommand\subsectionpostlude{\vspace{-\parskip}}
\RedeclareSectionCommands[
    beforeskip=.5\baselineskip,
    afterskip=0.25\baselineskip
]{section,subsection,subsubsection}


% code to define space before and after sections
\makeatletter
\let\origsubsection\subsection
\renewcommand\subsection{\@ifstar{\starsubsection}{\nostarsubsection}}
\newcommand\nostarsubsection[2][\relax]{
  \subsectionprelude
  \ifx\relax#1\origsubsection{#2}\else\origsubsection[#1]{#2}\fi
  \subsectionpostlude}
\newcommand\starsubsection[2][\relax]{
  \subsectionprelude
  \ifx\relax#1\origsubsection*{#2}\else\origsubsection*[#1]{#2}\fi
  \subsectionpostlude}
\makeatother


% Set the space between paragraphs and deactivate indentation for paragraphs
\setlength{\parskip}{1.4\baselineskip}
\setlength{\parindent}{0pt}
\fi

%-------------------------------------------
% Literal names referencing titles or card names (in a game): Capitalization in dark red
\newcommand{\cardref}[1]{{\textsc{\color{red}#1}}}

%-------------------------------------------
%CORE CONCEPTS (NO CARD NAMES) : SC in dark blue
\newcommand{\concept}[1]{\textsc{\lowercase{\color{blue}#1}}}



%-------------------------------------------
% reduce the space between each item of itemize lists
\setlist{nosep}

% Reduce the space before and after a list (the itemize environment)to 5pt 
\setlist[itemize]{parsep=5pt}

%-------------------------------------------
% lists
\setlist[itemize,1]{wide = 0pt,labelwidth = 0cm, label=\color{red}\ding{110}} 
\setlist[itemize,2]{wide=4pt,leftmargin=4pt, label=\color{red}\ding{121}} %list within a list  
%\setlist[itemize,3]{label={}} %list within a list within a list
%\setlist[itemize,4]{label={}} %list within a list within a list within a list


%-------------------------------------------
% numbered lists
% reduce the space between each item of enumerate lists
\setenumerate{nosep}


%-------------------------------------------
% Formatting of numbers in an enumerated list, replace #1 with the commented part to get circled numbers.
% You can also use the circled command for, for example, page numbers.
\newcommand*\circled[1]{#1}
%\tikz[baseline=(char.base)]{\node[shape=circle,fill=blue!20,draw,inner sep=2pt] (char) {#1};}}

\setenumerate[1]{wide = 0pt,labelwidth = 0cm, leftmargin =0cm, label=\protect\circled{\arabic*}}
\setenumerate[2]{labelwidth =.5cm, align = right, leftmargin =.5cm, label=\color{blue}\textbf{\arabic*.}}
\setenumerate[3]{labelwidth =1cm, align = right, leftmargin =1cm, label=\textbf{.\arabic*.}}


% Format listings (grey background)
\lstset{breaklines=true,backgroundcolor=\color{lightgray},tabsize=1,basicstyle=\ttfamily\footnotesize}

%-------------------------------------------
% Page background --- needs  bleed for print (check /lib/bookformat.tex and adapt page geometry)
\ifxetex
\AddToHook{shipout/background}[jinwen/opac]{
    \put (0in,-\paperheight){\includegraphics[width=\paperwidth,height=\paperheight]{images/white.png}}
}




%-------------------------------------------
% Page background pictures --- needs bleed for print

\newcommand{\cutlargepic}[1]{\AddToHookNext{shipout/background}{\put (0in,-22.8em){\includegraphics[width=\paperwidth]{#1}}}~\vspace{17em}}

\newcommand{\cutpic}[1]{\AddToHookNext{shipout/background}{\put (0in,-17.8em){\includegraphics[width=\paperwidth]{#1}}}~\vspace{14em}}
   
\newcommand{\cutbottompic}[1]{\AddToHookNext{shipout/background}{\put (0in,-\textheight){\includegraphics[width=\paperwidth]{#1}}}}    
\else

% For ebooks, just add them as pictures. Maybe even do not show the pictures for ebooks.

\newcommand{\cutlargepic}[1]{\includegraphics[width=\paperwidth]{#1}}
\newcommand{\cutpic}[1]{\includegraphics[width=\paperwidth]{#1}}
\newcommand{\cutbottompic}[1]{\includegraphics[width=\paperwidth]{#1}}
\fi

%-------------------------------------------
% Custom symbols that can be used as part of the text
\newcommand{\vcenteredinclude}[1]{\begingroup\setbox0=\hbox{\includegraphics[width=0.1in]{#1}}\parbox{\wd0}{\box0}\endgroup}
\newcommand{\squareIcon}{\vcenteredinclude{images/square.png}}


