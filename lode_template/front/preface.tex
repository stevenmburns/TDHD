%%%%%%%%%%%%%%%%%%%%%%%%%%%%%%%%%%%%%
% Read the /ReadMeFirst/ReadMeFirst.tex for an introduction. 
% Check out the accompanying book "Even Better Books with LaTeX the Agile Way in 2023" for a discussion of the template and step-by-step instructions. 
% The template was originally created by Clemens Lode, LODE Publishing (www.lode.de), 1/1/2023. Feel free to use this template for your book project!
% Contact me at mail@lode.de if you need help with the template or want editing and publishing services.
%%%%%%%%%%%%%%%%%%%%%%%%%%%%%%%%%%%%%

% The preface is written by the author.

\chapter{Preface}\label{preface:cha}

\begin{myquotation}
Feel free to add a quotation that describes your journey of writing the book as an author. Something personal is good.\end{myquotation}

Describe how you got the idea for writing the book and the personal journey of getting it from concept to creation. The reader should be able to understand why the book exists.

Also, give a short overview of what the book is about.

\noindent \textbf{\yourName}

\textbf{\yourCity, \yourCountry, \prefaceDate}



\hfil\psvectorian[height=10mm]{46}\hfil

% Define series in setup.tex if you want to display the previous part of the series.
\ifseries

\begin{center}
	
Check out the previous installment in this book series \bfseries \sffamily \LARGE \titleOfTheBookSeries\par:

\bfseries \Large PART \partPreviousPart: \titlePreviousPart\par
\psvectorian[height=8mm]{75}

~\\
\bfseries \small Published by \mypublishingcompany, \mypublishingcompanylocation\par

\ifxetex
    \includegraphics[width=0.5\textwidth]{\prevousCoverImageHiRes}
\else
    \includegraphics{\previousCoverImage}
\fi
\end{center}
\fi
